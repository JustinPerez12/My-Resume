%!TEX TS-program = xelatex
\documentclass[]{friggeri-cv}
\usepackage{afterpage}
\usepackage{hyperref}
\usepackage{color}
\usepackage{xcolor}
\hypersetup{
    pdftitle={},
    pdfauthor={},
    pdfsubject={},
    pdfkeywords={},
    colorlinks=false,       % no lik border color
   allbordercolors=white    % white border color for all
}
\addbibresource{bibliography.bib}
\RequirePackage{xcolor}
\definecolor{pblue}{HTML}{0395DE}

\begin{document}
\header{Justin}{Perez}
      {(Full Stack) Software Engineer}
      
% Fake text to add separator      
\fcolorbox{white}{gray}{\parbox{\dimexpr\textwidth-2\fboxsep-2\fboxrule}{%
.....
}}

% In the aside, each new line forces a line break
\begin{aside}
  \section{Phone}
    +7707780722
    ~
  \section{Mail}
    justinprz12@gmail.com
    ~
  \section{Web \& Git}
    \href{http://www.justindavidperez.com}{justindavidperez.com}
    \href{https://github.com/JustinPerez12}{github.com/justinprz12}
  \section{Programming Languages}
    Go, Java, Javascript, Typescript, Python
  \section{Frameworks and Libraries}
  React, React Native, Prisma, ASP.Net Core, Gluestack, Material UI, Tailwind CSS
  \section{Cloud and Devops}
    AWS Lambda, ECS Fargate, ECR, DynamoDB, CloudWatch, CodeBuild, S3, Neptune, Azure DevOps, Terraform, Azure Contianer Apps, Azure Contianer Repository
  \section{Soft Skills}
    Leadership, effective communication, and critical thinking
\end{aside}

\section{Experience}
\begin{entrylist}
	  \entry
	{06/2023 - Now}
	{Cox Automotive, Software Engineer}
	{Draper, UT}
	{ 
      Developed and maintained enterprise-level securtiy software used by hundres of software teams across the company. 
    \\Built reporting ETL jobs leveraging Cloudwatch Events, ECS Fargate, Lambda, and more. 
    \\Developed and maintained a front-end web application using React.js and CloudFront, paired with a C\# API hosted in Amazon ECS. The application enables users to monitor their security posture, customize team vulnerability settings, and track application vulnerabilities. 
    \\Developed and maintained a graph database, levering NeptuneDB, storing thousands of interconnected entities, including security vulnerabilities, software components, teams, and users, to streamline data accessibility and support security workflows. 
    \\Increased security adoption of overall company inventory at Cox Automotive from 10\% to over 45\% by creating simplified upload processes and efficient defect retrieval methods. 
    \\Designed and implemented tools that generate comprehensive security status reports for more than 7,500 software components.
    \\Integrated security solutions with industry-leading vendors, including Veracode, NoName, and Wiz.
    \\Enhanced the overall security workflow, improving the user experience for software teams and optimizing security practices across the organization.
  }
  \entry
    {11/2024 - Now}
    {CampusCore, Software Engineer}
    {Remote}
    {Responsible for feature work on hundreds of microservices. 
    \\Developed a mobile application using React Native and the Gluestack Component Library. 
    \\Built a backend system leveraging Node.js, seamlessly integrated with a PostgreSQL database hosted on Azure. Leveraged Prisma ORM to increase speed in development, and mitigate SQL injection risk. 
    \\Backed by Microsoft for Startups. 
    }
  \entry
    {03/22 - 01/2023}
    {Wavetronix, Software Engineering Intern}
    {Provo, UT}
    {Developed and maintained a C\# application that simulated traffic patterns, allowing users to design intersections and optimize layouts for traffic engineers. 
    \\Created a C\# application to process traffic simulation data, generating visual representations in graph form for easier data analysis and interpretation.
    \\Contributed to improving traffic design processes by making complex simulation data more accessible and understandable to engineers.
    }
\end{entrylist}

\section{Education}
\begin{entrylist}
  \entry
    {2019 - 2023}
    {Bachelor's Degree in Computer Science}
    {University of Utah}
    {}
\end{entrylist}

\end{document}
