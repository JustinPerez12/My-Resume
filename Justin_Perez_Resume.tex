%!TEX TS-program = xelatex
\documentclass[]{friggeri-cv}
\usepackage{afterpage}
\usepackage{hyperref}
\usepackage{color}
\usepackage{xcolor}
\hypersetup{
    pdftitle={},
    pdfauthor={},
    pdfsubject={},
    pdfkeywords={},
    colorlinks=false,       % no lik border color
   allbordercolors=white    % white border color for all
}
\addbibresource{bibliography.bib}
\RequirePackage{xcolor}
\definecolor{pblue}{HTML}{0395DE}

\begin{document}
\header{Justin}{Perez}
      {(Full Stack) Software Engineer}

% In the aside, each new line forces a line break
\begin{aside}
  \section{Contact}
    +1770-778-0722
    justinprz12@gmail.com
    \href{https://www.linkedin.com/in/justindperez/}{LinkedIn}
    ~
  \section{Web \& Git}
    \href{http://www.justindavidperez.com}{justindavidperez.com}
    \href{https://github.com/JustinPerez12}{github.com/justinprz12}
  \section{Programming Languages}
    Go, Java, Javascript, Typescript, Python, C\#
  \section{Frameworks and Libraries}
    React, React Native, Prisma, ASP.Net Core, Gluestack, Material UI, Tailwind CSS
  \section{Cloud and Devops}
    AWS Lambda, ECS Fargate, ECR, DynamoDB, CloudWatch, CodeBuild, S3, Neptune, Azure DevOps, Terraform, Azure Contianer Apps, Azure Contianer Repository
  \section{Soft Skills}
    Leadership, effective communication, and critical thinking
\end{aside}

\section{Professional Summary}
Versatile Software Engineer with over 2 years of experience designing, developing, and deploying scalable backend systems and cloud-based applications. Skilled in building secure microservices architectures, deploying APIs with Azure Container Apps (ACA) and Azure Container Registry (ACR), and managing PostgreSQL databases within virtual networks for strong security and seamless communication. Led development at CampusCore, creating a mobile app with React Native and a secure backend using Node.js and Azure. At Cox Automotive, developed enterprise security software, engineered ETL pipelines with AWS Lambda, and maintained a front-end platform using React.js and a C\# API on Amazon ECS, enhancing security monitoring for thousands of software components.
\section{Experience}
\begin{entrylist}
	  \entry
	{06/2023 - Now}
	{Cox Automotive, Software Engineer}
	{Draper, UT}
	{ 
    Increased security adoption across Cox Automotive's software inventory from 10\% to over 45\% by developing streamlined upload processes and efficient defect retrieval methods.
    Engineered and maintained enterprise-level security software used by hundreds of software teams, enhancing the overall security posture.
    Developed and maintained a React.js front-end application paired with a C\# API hosted on Amazon ECS, allowing users to monitor security status and manage team vulnerability settings effectively.
    Built reporting ETL jobs written in Python and Go leveraging AWS services such as Lambda, CloudWatch Events, and ECS Fargate to support comprehensive security metrics.
    Designed and maintained a graph database using NeptuneDB to streamline data accessibility across thousands of interconnected security entities.
    Integrated advanced security solutions with industry leaders including Veracode, NoName, and Wiz, enhancing the overall security workflow.
    Created tools for generating detailed security status reports for over 7,500 software components, improving security decision-making.
  }
  \entry
    {11/2024 - Now}
    {CampusCore, Software Engineer}
    {Remote}
    {
    Designed and developed a mobile application using React Native and the Gluestack Component Library, enhancing cross-platform usability.
    Built a scalable backend system using Node.js, integrated with a PostgreSQL database hosted on Azure, utilizing Prisma ORM to enhance development efficiency and reduce SQL injection risks.
    Spearheaded the deployment of all services to Azure, leveraging Azure Container Apps (ACA) and Azure Container Registry (ACR) to host the API effectively.
    Deployed the PostgreSQL database securely within a virtual network alongside other services, enabling seamless communication between services while maintaining a strong security posture.
    Led the development and deployment efforts for this architecture, ensuring robust security measures and efficient integration across microservices.
    Contributed to a startup initiative backed by Microsoft for Startups, focusing on innovative microservices solutions.
    }
  \entry
    {03/22 - 01/2023}
    {Wavetronix, Software Engineering Intern}
    {Provo, UT}
    {Developed a C\# application simulating traffic patterns, allowing engineers to design intersections optimized for traffic flow.
    Built a visualization tool in C\# to process traffic simulation data and generate insightful graphs, enhancing data interpretation and accessibility for traffic engineers.
    Assisted in improving traffic design processes by transforming complex simulation data into user-friendly formats.
    }
\end{entrylist}

\section{Education}
\begin{entrylist}
  \entry
    {2019 - 2023}
    {Bachelor's Degree in Computer Science}
    {University of Utah}
    {}
\end{entrylist}

\section{Projects}
\begin{entrylist}
  \entry
    {2022 - 2023}
    {Capstone Project - Huddle Up}
    {University of Utah}
    { Developed a group fantasy football web application focused on enhancing the team aspect of fantasy sports.
      Utilized React for building the frontend, with Tailwind CSS and Mantine for responsive and modern styling.
      Implemented a robust backend using Node.js, leveraging TypeScript for type safety and maintainable code.
      Designed and integrated a MySQL database using Prisma ORM for efficient data modeling and management.
      Managed version control and collaborative development using GitLab, ensuring seamless integration and deployment.}
\end{entrylist}
\end{document}

